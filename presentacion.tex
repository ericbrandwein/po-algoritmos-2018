\documentclass{beamer}

\usepackage[utf8]{inputenc}
\usepackage[spanish]{babel}
% \usepackage{comment}

\usetheme{Frankfurt}

%Information to be included in the title page:
\title{Prueba de Oposición}
\subtitle{Área Algoritmos}
\institute{Universidad de Buenos Aires, FCEyN}
\author{Eric Brandwein}
\date{6 de Noviembre de 2018}

\begin{document}

\frame{\titlepage}

\section{Marco del Ejercicio}
\begin{frame}
\frametitle{Marco del Ejercicio}
\large Materia: Métodos Numéricos
\pause
\\[12pt]
Guías prácticas:
\normalsize
\begin{itemize}
    \item 1: Elementos de álgebra lineal
    \item 2: Eliminación gaussiana / descomposición LU / normas y número de condición
    \item 3: Matrices simétricas definidas positivas / factorización de Cholesky
    \item \textbf{4: Matrices ortogonales / factorización QR}
    \item 5: Autovalores-autovectores / método de la potencia
    \item 6: Descomposición en valores singulares
    \item 7: Sistemas iterativos
    \item 8: Cuadrados mínimos lineales
\end{itemize}

\end{frame}


\begin{frame}
\frametitle{Conceptos necesarios}
Los alumnos deberían conocer los conceptos de:
\begin{itemize}
    \item Matriz \textbf{ortogonal} e \textbf{inversible}.
    \item Factorización \textbf{LU} y \textbf{QR}.
\end{itemize}
\end{frame}


\begin{frame}
\frametitle{Cuándo darlo}
El ejercicio puede ser dado en la clase
\textbf{anterior al parcial} como práctica para el mismo.
\end{frame}

\section{Objetivos del ejercicio}
\begin{frame}
\frametitle{Objetivos del ejercicio}
\begin{itemize}
    \item \textbf{Relacionar} los conceptos de factorización
        $LU$ con factorización $QR$.
    \item \textbf{Practicar} técnicas de resolución de
        ejercicios de este tipo para el parcial.
\end{itemize}
\end{frame}


\section{¿Por qué este ejercicio?}
\begin{frame}
\frametitle{¿Por qué este ejercicio?}
\begin{itemize}
    \item Integra \textbf{conceptos variados}, como ser el
        cálculo de determinantes,
        matrices transpuestas y factorizaciones de matrices.
        % También de matriz triangular y diagonal
    \item Su enunciado es \textbf{corto} y
        \textbf{fácil de entender}.
\end{itemize}

\end{frame}

\section{Enunciado}
\begin{frame}
\frametitle{Enunciado}
Sea \( A \in \mathbb{R}^{n \times n} \) una matriz inversible
y sea $A = QR$ una factorización $QR$ de $A$. Hallar una
factorización $LU$ ($L$ con unos en la diagonal) de $A^tA$.
\end{frame}

\section{}
\begin{frame}
\frametitle{¡Muchas Gracias!}
\centering
¿Preguntas?
\end{frame}

\end{document}
