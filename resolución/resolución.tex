\documentclass{amsart}
\usepackage[utf8]{inputenc}
\usepackage[spanish]{babel}
\usepackage{amssymb}

\begin{document}
$A = QR$, $Q$: ortogonal, $R$: triangular superior
\[
    A^tA = (QR)^t QR = R^tQ^tQR = R^tQ^{-1}QR = R^tIR = R^tR
\]

$R^t$ es diagonal inferior. Podemos decir que $R^t = R'^tD$,
con $D$ una diagonal tal que $R'^t$ tiene solo unos en la
diagonal. Si usamos $D$ como la diagonal de $R$, siempre
que $R$ no tenga ceros en la diagonal vamos
a poder encontrar a $R'$. Veamos que $R$ no puede tener ceros
en la diagonal.

$A$ es inversible, como dice en el enunciado. Por lo tanto,
\[
    det(A) \neq 0 \iff det(QR) = det(Q)det(R) \neq 0
    \iff det(Q) \neq 0 \land det(R) \neq 0
\]

Como R es una triangular superior, su determinante es igual
a la multiplicacion de los valores de su diagonal. Entonces,
ningún valor de su diagonal puede ser 0.

Ahora que encontramos $R'^t$, reescribiremos a $R^tR$ como
\(
    R^tR = R'^tDR
\). Como $R'^t$ es una triangular inferior con unos en la
diagonal, podemos tomarla como nuestra $L$, y como $DR$,
por ser multiplicación de triangulares superiores,
es triangular superior, podemos tomarla como nuestra $U$,
tal que $A^tA = R'^tDR = LU$. Luego, encontramos una
factorización $LU$ de $A^tA$, como pedía el enunciado.
$\hfill\blacksquare$

\end{document}
